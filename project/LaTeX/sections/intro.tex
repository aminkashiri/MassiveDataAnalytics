\section{مقدمه}
در این پروژه، با استفاده از ایده‌ها و روش‌های مختلفی که در طول ترم آموختیم، 
اطلاعاتی را از داده‌ها بیرون کشیدم. بخش‌های مختلف این پروژه را در فایل‌های 
jupyter
جداگانه قرار داده‌ام، و توضیحات هر بخش را نیز جداگانه در این فایل نوشته‌‌ام. 


\subsection{توضیحات کلی}

\begin{enumerate}
	\item 
	در ابتدای تمامی کدها تنظیمات اولیه اسپارک را انجام دادم، و سپس فایل 
	csv
	داده شده را 
	load
	کردم. 

	\item 
	در بعضی از بخش‌ها، روز ۸م را از داده‌ها حذف کردم. دلیل این کار این بود از تمامی روز‌ها
	به اندازه‌ی متناسب با هم داده داشته باشیم. در غیر این صورت تعداد داده‌ها از روز سه شنبه 
	دو برابر باقی روز‌ها می‌شد. کار‌های دیگری نیز می‌توانست انجام بگیرد. مثلا می‌شود داده‌های روز سه شنبه را 
	میانگین بگیریم (
		یعنی در تمام قسمت‌هایی که تعداد متغییری را شمر‌ده‌ایم، برای روز سه شنبه این تعداد را 
		تقسیم بر ۲ کنیم
	). 
	در بعضی از قسمت‌ها اما این زیادتر بودن داده‌های روز سه شنبه مشکلی ایجاد نمی‌کرد. اما دقت 
	کنید که در بعضی‌ از قسمت‌های دیگر می‌تواند تحلیل ما را دچار انحراف کند (
		مثلا ممکن است به اشتباه نتیجه‌ بگیریم که روز سه شنبه روز پر تردد تری است، یا 
		دوربین‌هایی که در روز سه شنبه دیده‌ می‌شوند را به اشتباه مهم تر در نظر بگیریم
	). 

	\item 
	توضیحات کد و روند اجرا را در فایل‌های 
	jupyter
	نوشته‌ام. سعی کرده‌ام که توضیحات منطق پشت کد‌ها را در این مستند بنویسم (
		و نه در خود کد‌ها
	). 
	بنابرین توضیحات تکنیکال خود کد در اینجا کمتر نوشته شده است. 
\end{enumerate}

