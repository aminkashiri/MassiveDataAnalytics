\section{بقیه ایده‌ها}

ایده‌ی دیگری که داشتم اما فرصت نکردم سراغ آن بروم، این بود که با استفاده از زمان‌های ثبت شده‌ی یک ماشین خاص در یک دوربین، 
پارک بودن آن ماشین را تشخیص دهم. برای مثال یک معیار ساده می‌توانست این باشد که در کمتر از یک ساعت، یک ماشین بیش از ۵ بار دریک دوربین دیده شود. 
با چنین روش‌هایی می‌شود نقاط قابل پارک کردن را پیدا کرد، و یا حتی دسته‌ای از ماشین‌ها که بیشتر پارک می‌کنند را جدا کرد. حتی می‌توان قصد ماشین‌های مختلف را نیز تشخیص داد (
    عبوری یا در حال پارک
). همچنین از الگوریتم‌های 
streaming
به دلیل کمبود وقت نتوانستم استفاده کنم. در قسمت‌هایی راجب پیدا کردن نقشه‌ی شهر صحبت کردم، که می‌توانستم به دنبال کتاب‌خانه‌های مختلفی بگردم تا 
شاید بتوانم نتیجه‌ای قابل نمایش به دست آوردم و با نقشه‌ی واقعی نقاطی که دوربین‌ها قرار دارد تطبیق دهم. در بخش اول نیز، دوربین‌های با اشتباه زیاد را 
پیدا کردیم. کار دیگری که می‌توانستیم انجام دهیم این بود که تنها دوربین‌های با خطای پایین را در نظر بگیریم، و آن وقت شاید 
دقت بعضی از بخش‌های قبلی بهتر می‌شد. 