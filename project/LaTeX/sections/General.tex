\section{General}

در اولین بخش از پروژه، سعی کردم اطلاعات اولیه‌ای را از داده بیرون بکشم، تا در 
بخش‌های بعدی بهتر بتوانم تحلیل کنم. کارهایی که در این بخش انجام شده است:

\begin{enumerate}
    \item 
    تعداد کل داده‌ها، کل ماشین‌ها و کل دوربین‌ها را به دست آوردم. این مقادیر 
    به ترتیب برابر با 
    34989160
    و 
    5487645
    و 
    1035
    بودند.

    \item 
    پر تکرارترین ماشین‌ها، و پر تکرارترین ترکیب ماشین و دوربین را به دست آوردم. 
    در این مرحله متوجه شدم که یکی از ماشین‌ها داده‌ی پرت است (
        میزان ثبت شدن آن بیش از ۱ بار در هر ثانیه است
    ). به همین دلیل 
    در بخش‌های بعدی این داده را حذف کردم. دقت کنید که حتی اگر این داده‌پرت نباشد، 
    و مثلا ثبت شدن‌های متوالی یک ماشین پارک شده باشد، در قسمت‌های بعدی 
    تحلیل ما را با مشکل مواجه می‌کند. پس از نظر مفهومی حذف آن واجب به نظر می‌رسید.

    \item 
    خلاصه‌ای از 
    data
    را به دست آوردم و نمایش دادم، مانند میانگین، کمینه، بیشینه، چارک اول و سوم و ...

    \item 
    تنها ردیف‌هایی که یک تخلف بودند را نگه داشتم، و پس از حذف داده‌های پرت، متخلف‌ترین 
    ماشین‌ها را نمایش دادم. 
    بیشترین تخلف مربوط به ماشین
    73138295
    بود با 
    730
    مورد تخلف.

    \item 
    نرخ خطای دوربین‌ها را به کمک دو ستون
    \lr{ORIGINE\underline{ }CAR\underline{ }KEY}
    و
    \lr{FINAL\underline{ }CAR\underline{ }KEY}
    به دست آوردم (اختلاف این دو نشان‌ دهنده‌ی اشتباه است) و سپس به کمک آن دوربین‌ها را بر اساس دقت 
    مرتب کردم و دقت با‌دقت‌ترین و کم‌دقت ترین‌ آن‌ها را به دست آوردم.
\end{enumerate}